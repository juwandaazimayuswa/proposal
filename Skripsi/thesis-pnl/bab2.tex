\begin{landscape}
\chapter{TINJAUAN PUSTAKA}
\section{\textit{State of the Art}}
Penyusunan penelitian ini mengambil beberapa referensi terdahulu yang diperoleh dari artikel yang telah dipublikasi melalui jurnal-jurnal yang sudah ber-ISSN dan mempunyai hubungan dengan penelitian yang akan dilakukan.


Berikut beberapa \textit{State of the Art} terdapat pada tabel 2.1.

\begin{center}
\begin{longtable}{| c | L{3cm} | L{4cm} | L{3cm} | L{5cm} | L{2.5cm} | L{4cm} |}
\caption{Paparan \textit{State of the Art}}
\label{tb:stateoftheart} \\

\hline
No &
Penulis/Tahun &
\multicolumn{1}{c|}{Judul Artikel} &
\multicolumn{1}{c|}{Metode} &
\multicolumn{1}{c|}{Hasil yang diperoleh} &
\multicolumn{1}{c|}{Persamaan} &
\multicolumn{1}{c|}{Perbedaan} \\ \hline
\endfirsthead


\hline
No &
Penulis/Tahun &
\multicolumn{1}{c|}{Judul Artikel} &
\multicolumn{1}{c|}{Metode} &
\multicolumn{1}{c|}{Hasil yang diperoleh} &
\multicolumn{1}{c|}{Persamaan} &
\multicolumn{1}{c|}{Perbedaan} \\ \hline
\endhead




1 	& Tri Mulyono, Kusworo Adi dan Rahmat Gernowo /2012
 	& SISTEM PENGENALAN WAJAH DENGAN METODE \emph{EIGENFACE} DAN JARINGAN SYARAF TIRUAN (JST)
 	& \emph{Eigenface}
 	& Pengenalan wajah dengan algoritma \emph{eigenface} dapat digunakan untuk mengidentifikasi wajah meskipun objek dengan ekspresi wajah yang berbeda.
 	& Menggunakan metode \emph{Eigenface}
 	& Pada penelitian ini hanya melakukan pengujian metode yang digunakan, tidak dikembangkan menjadi sebuah aplikasi
 	\\ \hline
2 	& Cahya Rahmad, Kadek Suarjuna Batubulan, Syafri Wira Wicaksana /2019
 	& Absensi Kelas Otomatis Melalui Pengenalan Citra Wajah Menggunakan Metode \emph{Principal Component Analysis}
 	& \emph{Eigenface} 
 	& Dari keseluruhan pengujian pengenalan wajah
menggunakan metode \emph{principal component analysis} yang dilakukan didapatkan hasil keakurasian sistem sebesar 76.
 	& Menggunakan metode yang sama dan melakukan penelitian tentang Absensi
 	& Penelitian hanya melakukan pengujian menggunakan web, sedangkan yang peneliti lakukan adalah membangun sebuah aplikasi. 
 	\\ \hline
3 	& Mohammad Arya Rosyd Sikumbang, Roni Habibi, Syafrial Fachri Pane /2020
 	& Sistem Informasi Absensi Pegawai Menggunakan Metode RAD dan Metode LBS Pada Koordinat Absensi
 	& \emph{Rapid Application Development} (RAD)
 	& Pegawai yang sedang melakukan dinas luar dikantor sudah bisa melakukan absensi tanpa harus kekantor terlebih dahulu dan beban kerja yang diterima oleh pegawai sedikit berkurang.
 	& Melakukan penelitian tentang aplikasi absensi 
 	& Sistem informasi ini hanya dibangun berbasis \emph{website} dan \emph{nge-build web} ke apk \emph{android,} sehingga di harapkan ke depannya dapat dikembangkan menjadi lebih baik lagi.
 	\\ \hline
4 	& Andi Maulidinnawati Abdul Kadir Parewe, A. Sumardin, Muhammad Isra Pratama /2022
 	& Penerapan Metode \emph{Scrum} dengan \emph{Framework} Flutter dalam Teknologi \emph{Location based service} Pada Sistem Provos Polisi
 	& \emph{Scrum}
 	& Berdasarkan hasil pengujian \emph{Blackbox, admin/user}
dapat melihat lokasi dari masing-masing personil pada halaman Maps dan titik koordinat \emph{user} akan terupadate secara otomatis setiap kali \emph{user} bergerak sejauh 100 meter. Dan berdasarkan pengujian UAT, aplikasi Daeng Provos sangat disarankan untuk diimplementasikan pada Polsek Panakkukang.
 	& Melakukan penelitian tentang aplikasi absensi menggunakan \emph{framework} flutter
 	& Pada penelitian ini menggunakan metode \emph{Scrum}, pada penelitian yang akan penguji lakukan menggunakan metode \emph{Eigenface}
 	\\ \hline
5 	& I Nyoman Tri Anindia Putra , Ida Bagus Gede Dwidasmara ,
I Gede Santi Astawa /2014
 	& PERANCANGAN DAN PENGEMBANGAN SISTEM ABSENSI REALTIME MELALUI METODE PENGENALAN WAJAH
 	& \emph{Eigenface}
 	& Dapat disimpulkan tingkat akurasi dari sistem ini dalam pendeteksian wajah adalah jumlah percobaan berhasil/jumlah percobaan *100 Sehingga diperoleh hasil dari 9/10*100 = 90
 	& Melakukan penelitian tentang aplikasi absensi.
 	& Melakukan penelitian tentang aplikasi absensi menggunakan bahasa pemrograman \emph{CSharp}.
 	\\ \hline
6	& Susi Tamba /2022
 	& Perancangan Aplikasi Absensi Karyawan Dengan Deteksi Wajah Menggunakan Metode \emph{Eigenface}
 	& \emph{Eigenface}
 	& Penerapan metode \emph{eigenface} untuk mendeteksi wajah dengan membandingkan nilai matriks dari wajah kemudian menormalisasikannya sampai perhitungan \emph{weight} dan jarak dari wajah (\emph{image}) tersebut mendekati dan dapat ditemukan.
 	& Melakukan penelitian tentang aplikasi absensi.
 	& Melakukan penelitian tentang aplikasi absensi menggunakan \emph{Mathlab}.
 	\\ \hline
7	& Munawir, Liza Fitria, Muhammad Hermansyah /2020
 	& Implementasi \emph{Face Recognition} pada Absensi Kehadiran Mahasiswa Menggunakan Metode \emph{Haar Cascade Classifier}
 	& \emph{Haar Cascade Classifier}
 	& Tingkat akurasi implementasi \emph{face recognition} pada absensi kehadiran mahasiswa menggunakan metode \emph{Haar Cascade Classifier} dengan pengujian satu wajah adalah 76 dan pengujian banyak wajah adalah 33.33.
 	& Melakukan penelitian tentang aplikasi absensi.
 	& Melakukan penelitian tentang aplikasi absensi menggunakan metode \emph{Haar Cascade Classifier}.
 	\\ \hline
		 	 	
\end{longtable}
\end{center}


\end{landscape}

\section{Tinjauan Teoritis}
\subsection{Flutter}
\emph{Flutter} adalah \emph{framework} yang digunakan oleh para pengembang untuk membangun aplikasi berbasis \emph{mobile multiplatform}. Sehingga dalam sekali pengkodean, aplikasi yang telah di-\emph{build} dapat dijalankan untuk membuat antarmuka yang cantik untuk aplikasi mobile yaitu android dan iOS dan Bahasa program yang digunakan yaitu \emph{framework flutte} dan dikenal juga sebagai \emph{dart} dimana \emph{dart} ialah bahasa pemrograman yang dikembangkan oleh \emph{Google} untuk kebutuhan umum. \emph{Dart} dapat digunakan pada beberapa platform diantaranya \emph{flutter,} web dan server.

\emph{Framework Flutter} sendiri merupakan \emph{framework open-source} yang populer digunakan untuk membangun aplikasi \emph{mobile}, desktop, dan web. \emph{Framework} ini memiliki keunggulan dalam hal fleksibilitas dan kecepatan pembangunan aplikasi, sehingga menjadi pilihan yang menarik untuk membangun aplikasi presensi dengan metode pengenalan wajah. \cite{and}

\subsection{Visual Studio}
\emph{Microsoft visual studio} merupakan sebuah IDE \emph{(Intregrated Development Environment)} dari Microsoft untuk pengembangan aplikasi. IDE sendiri merupakan program komputer yang memilik fasilitas yang diperlukan untuk pembangunan perangkat lunak. Dengan aplikasi visual studio, bisa membangun aplikasi GUI, aplikasi konsole, aplikasi web, maupun aplikasi mobile. \cite{k}

\subsection{Use Case Diagram}
Diagram yang memperlihatkan komponen use case dan aktor yang ada didalamnya, diagram ini berperan dalam memodelkan prilaku dari sistem yang dibutuhkan pengguna. \cite{ara}

\subsection{Android}
Android adalah sistem operasi \emph{mobile} yang dikembangkan oleh \emph{Google}. Android dirancang untuk digunakan pada berbagai perangkat mobile, seperti \emph{smartphone}, tablet, dan perangkat \emph{wearable}. Android menggunakan bahasa pemrograman \emph{Java} dan \emph{Kotlin}, dan menyediakan antarmuka pengguna yang intuitif, dukungan aplikasi yang luas, serta kemampuan untuk menghubungkan perangkat ke internet dan sumber daya lainnya.
Android menjadi salah satu sistem operasi mobile yang paling populer di dunia, dengan lebih dari 2 miliar perangkat aktif yang menggunakan Android pada tahun 2021. Android memiliki toko aplikasi sendiri yang disebut \emph{Google Play Store}, di mana pengguna dapat mengunduh aplikasi dan game untuk digunakan pada perangkat mereka. Selain itu, Android juga mendukung pengembangan aplikasi \emph{open-source}, sehingga para pengembang dapat membuat aplikasi untuk Android dan mempublikasikannya secara bebas. \cite{l}

\subsection{Android Studio}
\emph{Android Studio} adalah IDE yang sangat populer dan digunakan oleh para pengembang Android. IDE ini dirilis oleh Google pada tahun 2013 sebagai pengganti dari \emph{Eclipse} yang sebelumnya menjadi IDE pilihan pengembang Android.
Salah satu fitur paling penting dari \emph{Android Studio} adalah editor kode yang canggih. Editor ini dilengkapi dengan fitur seperti kode warna, navigasi kode yang mudah, dan integrasi dengan sistem kontrol versi seperti Git. Selain itu, Android Studio juga memiliki fitur \emph{refactor} yang dapat membantu mengubah kode secara cepat dan aman.
Selain editor kode, \emph{Android Studio} juga memiliki \emph{debugger} yang canggih.\emph{Debugger} ini memungkinkan pengembang untuk melacak \emph{bug} dan memperbaikinya dengan cepat. Selain itu, \emph{Android Studio} juga dilengkapi dengan emulator yang dapat digunakan untuk menguji aplikasi di berbagai perangkat Android. \cite{dims}

\subsection{XAMPP}
XAMPP adalah perangkat lunak bebas, yang mendukung banyak sistem operasi, merupakan kompilasi dari beberapa program. Fungsinya adalah sebagai server yang berdiri sendiri \emph{(localhost)}, yang terdiri atas program \emph{Apache HTTP Server, MySQL database}, dan penerjemah bahasa yang ditulis dengan bahasa pemrograman PHP dan Perl. \cite{faz}

\subsection{Docker}
\emph{Docker} dapat digunakan sebagai server dengan cara menjalankan aplikasi dan layanan dalam kontainer \emph{Docker}. Penggunaan \emph{Docker} sebagai server memberikan beberapa manfaat, antara lain:
\begin{enumerate}
\item Isolasi: \emph{Docker} memungkinkan pengguna mengisolasi aplikasi dan layanan dalam kontainer. Setiap kontainer memiliki lingkungan yang terisolasi, memungkinkan aplikasi berjalan tanpa interferensi dengan aplikasi lain atau sistem operasi \emph{host}. Dengan isolasi ini, pengguna dapat menjalankan berbagai aplikasi dan layanan dalam kontainer yang berbeda tanpa adanya konflik atau interdependensi.
\item Portabilitas: Kontainer \emph{Docker} adalah unit portabel yang berisi semua dependensi dan konfigurasi yang diperlukan. Pengguna dapat membuat kontainer di mesin pengembangan dan menjalankannya di mesin produksi dengan sedikit atau tanpa modifikasi. Ini mempermudah dalam deployment dan migrasi aplikasi ke lingkungan yang berbeda.
\item Skalabilitas: \emph{Docker} memungkinkan pengguna dengan mudah mengelola dan mengatur skala aplikasi. Pengguna dapat menyalin dan menjalankan lebih banyak kontainer dari image yang sama untuk meningkatkan \emph{throughput} dan ketersediaan aplikasi. \emph{Docker} juga mendukung orkestrasi kontainer yang lebih canggih melalui alat seperti \emph{Docker Swarm} atau Kubernetes, yang memungkinkan manajemen dan penyebaran kontainer secara terdistribusi.
\item Efisiensi sumber daya: \emph{Docker} menggunakan teknologi kontainerisasi yang ringan. Dibandingkan dengan virtualisasi tradisional, di mana setiap mesin virtual memiliki sistem operasi yang lengkap, \emph{Docker} menggunakan fitur-fitur kernel Linux yang ada untuk berbagi sumber daya dan mengurangi \emph{overhead}. Hal ini memungkinkan pengguna menjalankan lebih banyak kontainer dalam jumlah yang sama dengan infrastruktur yang lebih sedikit.
\item Reproduksi lingkungan: Dengan \emph{Docker}, pengguna dapat mereproduksi lingkungan secara konsisten, baik itu di mesin lokal, server, atau \emph{cloud}. Dengan menggunakan \emph{Dockerfile}, pengguna dapat mendefinisikan konfigurasi dan dependensi yang diperlukan untuk membangun \emph{image}, sehingga memastikan bahwa server yang dibuat berada dalam kondisi yang konsisten.\cite{sten}
\end{enumerate} 


\subsection{Black Box}
\emph{Black box testing} adalah salah satu jenis pengujian perangkat lunak yang dilakukan tanpa mengetahui rincian implementasi internal dari aplikasi yang sedang diuji. Dalam pengujian aplikasi, \emph{black box testing} sering digunakan untuk memeriksa fungsionalitas dan kinerja dari aplikasi, tanpa memeriksa kode sumber atau detail teknis lainnya.
\emph{Black box testing} biasanya melibatkan pengujian aplikasi dengan memberikan \emph{input} yang berbeda dan mengamati \emph{output} yang dihasilkan, untuk memastikan bahwa aplikasi berfungsi dengan benar dan sesuai dengan spesifikasi. Kegunaan \emph{black box testing} dalam pengujian aplikasi adalah membantu menemukan \emph{bug} atau masalah lain yang mungkin terjadi pada aplikasi, sehingga dapat diperbaiki sebelum aplikasi dirilis ke pengguna akhir. \cite{jem}