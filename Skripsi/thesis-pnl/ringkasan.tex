\chapter*{RINGKASAN}
\addcontentsline{toc}{chapter}{RINGKASAN}
Didalam perkembangan terknologi yang pesat, dapat dipastikan merambah ke dunia pendidikan, dalam media pembelajaran, interaksi antar guru dan murid, kegiatan praktek atau laboratorium dan lain sebagainya. Namun proses presensi pada beberapa sekolah masih menggunakan cara manual yang rawan akan kecurangan. Pengembangan aplikasi absensi akan menjadi suatu solusi untuk mempermudah proses absensi. Penggunaan metode \emph{face recognition} pada aplikasi absensi menjadikan proses absensi yang jauh lebih akurat dan efisien, sekaligus mempermudah absensi pada dunia pendidikan. Hipotesis dari penelitian ini adalah aplikasi presensi dengan \emph{face recognition} dan \emph{Machine learning} dapat membantu meningkatkan efisiensi dan akurasi serta mempermudah proses presensi dan mengurangi kecurangan pada presensi manual.  %\cite{tri} 

Kata kunci : Presensi, \emph{Face Recognition}, \emph{Machine learning} Mobile.