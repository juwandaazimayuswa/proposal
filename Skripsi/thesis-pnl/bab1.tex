\chapter{PENDAHULUAN}
\section{Latar Belakang Masalah}
Presensi menjadi suatu proses penting yang biasa dilakukan dalam kegiatan belajar mengajar, kerja, dan berbagai aktivitas lainnya. Pada SMK Negeri 1 Karang Baru proses presensi masih dengan cara manual, proses presensi manual yang dilakukan dengan menggunakan daftar hadir atau buku presensi sering kali rentan terhadap kecurangan dan tidak efisien. Perlu dilakukan pengembangan teknologi presensi yang lebih akurat, efisien, dan otomatis. Dalam beberapa tahun terakhir, pengolahan citra dan \emph{machine learning} telah menjadi topik penelitian yang menarik untuk pengembangan aplikasi presensi otomatis. Salah satu metode yang umum digunakan adalah metode pengenalan wajah, dimana pengolahan citra digunakan untuk membandingkan fitur wajah pada foto dengan \emph{database} wajah yang telah tersimpan sebelumnya. Metode ini memiliki keuntungan karena tidak memerlukan alat khusus seperti sidik jari atau kartu.\cite{nix}

Dalam konteks tersebut, maka dibuatlah aplikasi presensi dengan menggunakan metode \emph{Face Recogniton} berbasis mobile dan \emph{cloud computing}, untuk memperbaiki efisiensi dan akurasi presensi pada SMK NEGERI 1 KARANG BARU. Melalui penelitian ini diharapkan dapat memberikan solusi praktis bagi masalah presensi yang sering dihadapi oleh berbagai lembaga atau instansi.
\

\section{Rumusan Masalah}
Berdasarkan latar belakang di atas, rumusan masalah dalam penelitian ini dapat dirumuskan sebagai berikut:

\begin{enumerate}
\item Berapa kecepatan waktu server mengambil data didalam \emph{database} ?\
\item Bagaimana penerapan metode \emph{Face Recognition} pada aplikasi presensi ?
\end{enumerate}	

\section{Tujuan Penelitian}
Berdasarkan rumusan masalah yang telah dijabarkan sebelumnya, tujuan dari penelitian ini adalah sebagai berikut:

\begin{enumerate}
\item Mengetahui kecepatan waktu server dalam mengambil data didalam \emph{database}. \
\item Menghasilkan rancangan aplikasi presensi. \
\item Melihat akurasi keberhasilan menggunakan metode \emph{Face Recognition} pada aplikasi presensi.\
\end{enumerate}


\section{Batasan Masalah}
Dalam penelitian ini, terdapat beberapa batasan masalah yang harus diperhatikan, yaitu:
\begin{enumerate}
\item Penelitian ini memfokuskan pada pengembangan aplikasi absensi dengan menggunakan metode \emph{Face Recognition}.
\item Di batasi pada pendataan data siswa dan guru, daftar hadir serta laporan kehadiran siswa.
\item Melakukan survey pengujian pada 10 siswa yang ada di SMK Negeri 1 Karang Baru.
\item \emph{Cloud computing]} menggunakan \emph{docker}.
\end{enumerate}

\section{Manfaat Penelitian}
Beberapa manfaat dari penelitian ini antara lain:

\begin{enumerate}
\item Memberikan solusi dan kemudahan dalam pengembangan sistem presensi yang lebih efisien dan efektif dengan menggunakan teknologi \emph{face recognition}.
\item Meningkatkan kualitas penggunaan teknologi dalam dunia pendidikan dengan penggunaan sistem presensi yang lebih modern.
\item Meningkatkan efisiensi dan produktivitas dalam melakukan presensi pada SMK NEGERI 1 KARANG BARU, dengan tidak lagi memerlukan waktu yang lama untuk melakukan presensi secara manual.
\end{enumerate}